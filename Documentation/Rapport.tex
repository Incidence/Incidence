\documentclass[a4paper]{article}
\usepackage[utf8]{inputenc}  % si utf8
\usepackage{graphicx}
\usepackage{verbatim}

\pagestyle{plain}

\title{\textbf{Rapport de projet}\\- \Huge{Incidence} -}
\author{\emph{CHAMBONNET Kevin}\\\emph{GAUTHIER Silvère}\\\emph{MARTINEZ Thierry}\\\emph{MOKHRETAR Amin}}
\date{\today}

\newcommand{\alinea}{\hspace*{0.5cm}}

\begin{document}
  \maketitle
  \newpage
  \tableofcontents

  \newpage
  \part{Remerciements}

  \newpage
  \part{Cahier des charges}
  
	\section{Introduction}
	
	\section{Mecanismes de jeu}
	
	\section{Structure du jeu}
	
		\subsection{Moteur}
		
			\subsubsection{Moteur multi-agent}
			
			\subsubsection{Moteur de carte}
			
		\subsection{Scripts}
		
	\section{Elements graphiques}
	
	\section{Elements sonores}
	

  \newpage
  \part{Gestion du projet}

  \newpage
  \part{Développement}
  
	\section{Moteur de jeu}
	
		\subsection{Gestion des états}
		
		\subsection{Gestion de l'interface utilisateur}
		
		\subsection{Gestion des ressources et animations}
		
	\section{Moteur de carte}
	
		\subsection{Météo}
		
		\subsection{Incidences}
		
	\section{Moteur multi-agent}
	
		\subsection{Gestion des entités}
		
			\subsubsection{Santé}
			
			\subsubsection{Incidences}
			
		\subsection{Gestion des scripts}
		
		\subsection{Liste des scripts}
		
	\section{Architecture}
	
		\subsection{Machine à états}
			pour chaque etat :
			interface + graphiques + sons + action possibles
	

  \newpage
  \part{Post-Mortem}
  
	\section{Réalisations non abouties}
	
	\section{Améliorations réalisables}
	

\end{document}
