\documentclass[a4paper]{article}
\usepackage[utf8x]{inputenc}  % si utf8
\usepackage{graphicx}

\pagestyle{plain}

\title{\textbf{Technical Design Document}\\- \Huge{Incidence} -}
\author{\emph{CHAMBONNET Kevin}\\\emph{GAUTHIER Silvère}\\\emph{MARTINEZ Thierry}\\\emph{MOKHRETAR Amin}}
\date{\today}

\newcommand{\alinea}{\hspace*{0.5cm}}

\begin{document}
  \maketitle
  \newpage
  \tableofcontents

% Liens utiles : 
% https://www.digipen.edu/fileadmin/website_data/gallery/game_websites/NarbacularDrop/documents/narbacular_drop_technical_design_document.pdf

  \newpage
  \part{Organisation}
    \section{Gestion du projet}
      \subsection{Gestion de l'équipe}
        \alinea Tous les membres se connaissant et étant supposés être capable de travailler en équipe, nous n'avons fait aucune élection de chef de projet.\\
        \alinea Nous avons opté pour travailler de manière collégiale, et ainsi garder une cohésion de groupe sans pour autant avoir de hiérarchie instaurée au sein du groupe, qui pourrait au contraire déservir la réalisation de nos objectifs.\\
        \alinea Chaque membre a donc autant de pouvoir que les autres, et peut donc participer activement au projet, autant lors de la conception que du développement. Toutes les décisions seront prises suivant la majorité lors de votes.\\\\
        \alinea Pour ce qui est des réunions de projets, nous avons convenu avec notre tuteur d'une réunion, allant d'environ trente minutes à une heure, toutes les une à deux semaines, afin de mettre au point l'avancement du projet. En parallèle, tous les membres de notre équipe se retrouvent une fois par semaine afin de discuter des points clés effectués ou à venir, donner lieu aux votes pour les prises de décisions, ou encore, lors de la phase de développement, travailler en collaboration afin d'optimiser notre travail.\\\\
        \alinea Au niveau du travail collaboratif, nous avons mis en place un dépôt sur github, contenant tant la documentation telle que ce rapport que les sources de notre jeu. Par ailleurs, nous mettrons sur ce dépôt uniquement les fichiers sources, les images et les sons, mais en aucun cas les fichiers temporaires ou les exécutables. Un fichier "makefile" sera disponible pour quiconque voudrait compiler le programme chez lui. Les seuls fichiers binaires disponibles seront les PDF de la documentation, pour un soucis de facilité d'accès.

      \subsection{Découpage en tâches}
        \alinea Afin de préparer le développement du jeu, il était nécessaire de séparer les fonctionnalités les unes des autres. Nous avons abouti à ce diagramme, qui résume notre choix de découpage :
        \begin{center}
          \includegraphics[scale=0.5]{DiagrammeDecoupageProjet.png}
        \end{center}

      \subsection{Assignation}
        \alinea Le projet étant maintenant découpé en un certain nombre de modules, il ne restait plus qu'à assigner chaque tâche à un ou plusieurs membres de l'équipe. Nous nous sommes organisés comme ceci :
        \begin{itemize}
          \item \textbf{Scripts de l'IA :} MARTINEZ Thierry, MOKHRETAR Amin.
          \item \textbf{Moteur :}
          \begin{itemize}
            \item \textbf{Gestion de la carte :} GAUTHIER Silvère.
            \item \textbf{Gestion des entités :} CHAMBONNET Kevin.
          \end{itemize}
          \item \textbf{Interface :} Tous les membres.
        \end{itemize}
        \alinea Bien entendu, cette répartition n'est pas totalement fixée, elle concerne en réalité l'affectation de responsables de parties, qui seront en charge de celle-ci mais pourront évidemment faire appel aux autres membres pour trouver une solution à un problème par exemple.\\
        \alinea Le détail complet des tâches et assignations se situe dans la section Gestion du temps, page \pageref{GestionTps}.

      \subsection{Gestion du temps}
        \label{GestionTps}
        \alinea Afin de clarifier notre gestion du temps, un diagramme Gantt est disponible dans la documentation de notre projet, et sera mis à jour en fonction de l'avancée du projet.\\
        \alinea Voici tout de même une première estimation du temps nécessaire :
        \begin{center}
          %\includegraphics[scale=0.5]{DiagrammeGantt.png}
        \end{center}

    \section{Choix technologiques}
      \subsection{Langages de programmation}
        \alinea Pour des besoins de performances, nous avons comparé différents langages. Pour réduire le temps de recherche et de comparaison, nous nous sommes appuyé sur des tests déjà effectués par d'autre.\\
        \alinea Voici des tests de performances concernant un large panel de langages, comparés ici dans quatre contextes différents :\\
        \begin{center}
          \includegraphics[scale=0.5]{AnalyseLangage1.png}
          \includegraphics[scale=0.5]{AnalyseLangage2.png} 
        \end{center}
        \alinea Nous pouvons observer que globalement, le langage le plus rapide est ici C++. L'utilisation de ce langage étant très fréquente dans les jeux vidéos, de part sa réputation d'un des langages les plus performants, et tous les membres de notre équipe sachant l'utiliser, nous avons fait le choix de programmer le moteur du jeu en C++.\\
        \alinea Afin d'optimiser encore la rapidité du moteur, nous avons cherché à associer son coeur écrit en C++ avec un langage de scripting qui permettra de mettre en place les différentes actions du jeu.\\ D'après les graphiques ci-dessus, nous avons opté pour le langage LUA, performant et facile d'utilisation (syntaxe proche du C++). En effet, même si Python est très prisé et offre beaucoup plus de possibilités que LUA, nous l'avons estimé bien trop lourd pour l'utilisation que nous allons en faire.\\
        \alinea Les deux langages C++ et LUA sont souvent associés dans les jeux vidéos, notre choix suit donc la tendance, ce qui nous offre une certaine assurance.

      \subsection{Bibliothèques}
        \alinea Pour la gestion graphique des 150x150 tuiles composant la carte et des différentes entités, nous avons cherché une bibliothèque relativement simple d'utilisation mais surtout performante afin de garder la fluidité gagnée avec le choix des langages de programmation.\\
        \alinea Connaissant la bibliothèque OpenGL, qui est bas niveau et performante dans les affichages deux et trois dimensions, nous nous sommes tournés vers deux bibliothèques utilisant OpenGL : SDL et SFML.\\
        \alinea D'après plusieurs sites web et forums, les dernières versions (respectivement 2.0 et 2.1) de ces deux bibliothèques se valent en terme de performance.\\
        \alinea En confrontant nos préférances personnelles quant au choix de l'une ou l'autre, nous nous sommes finalement mis d'accord pour utiliser la bibliothèque graphique SFML 2.1, qui paraît plus simple d'utilisation que la SDL. De plus, elle permet une gestion aisée des fichiers son, ce qui sera un plus pour la finalité de notre jeu.

\end{document}
